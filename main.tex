\documentclass[12pt,letterpaper]{article}
\usepackage[utf8]{inputenc}

\usepackage{graphicx}

\setlength{\parskip}{1em}
\usepackage{hyperref}
\hypersetup{
    colorlinks=true,
    linkcolor=blue,
    filecolor=magenta,      
    urlcolor=cyan,
    pdftitle={Overleaf Example},
    pdfpagemode=FullScreen,
    }
\setlength{\parindent}{0em}

\begin{document}
\title{
\textbf{National Institute Of Technology, Raipur\\ 
 \\ \includegraphics[width = 4cm, ]{National_Institute_of_Technology,_Raipur_Logo .png}
\\ \\
DATA MINING}
}

\author{Dubi Sao, 19111025 
Artificial Intelligence ,\\ VI Semester,
        Biomedical Engineering Department,\\
        }
\date{}
\maketitle
\begin{flushright}
    Guided by :\\
    Dr. Saurabh Gupta
\end{flushright}
\rule{\textwidth}{1pt}
\begin{abstract}



\end{abstract}
\rule{\textwidth}{1pt}
\newpage
\textbf{TABLE OF CONTENT}\\
\rule{\textwidth}{1pt}
\tableofcontents
\rule{\textwidth}{1pt}

\section{Introduction}
The process of detecting anomalies, patterns, and correlations within massive databases in order to forecast future outcomes is known as data mining. This is accomplished by combining three fields that are intertwined: statistics, artificial intelligence, and machine learning. Data mining is an automated technique that involves looking for patterns in vast information that humans might miss.


Data mining approaches, for example, are used in weather forecasting. Weather forecasting involves analysing large amounts of past data to find trends and forecast future weather conditions based on the time of year, climate, and other factors.



As a result of this analysis, algorithms or models are developed that collect and analyse data in order to anticipate outcomes with increasing accuracy.

\end{document}
