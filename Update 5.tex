\documentclass[12pt,letterpaper]{article}
\usepackage[utf8]{inputenc}

\usepackage{graphicx}

\setlength{\parskip}{1em}
\usepackage{hyperref}
\hypersetup{
    colorlinks=true,
    linkcolor=blue,
    filecolor=magenta,      
    urlcolor=cyan,
    pdftitle={Overleaf Example},
    pdfpagemode=FullScreen,
    }
\setlength{\parindent}{0em}

\begin{document}
\title{
\textbf{National Institute Of Technology, Raipur\\ 
 \\ \includegraphics[width = 4cm, ]{National_Institute_of_Technology,_Raipur_Logo .png}
\\ \\
DATA MINING}
}

\author{Dubi Sao, 19111025 
Artificial Intelligence ,\\ VI Semester,
        Biomedical Engineering Department,\\
        }
\date{}
\maketitle
\begin{flushright}
    Guided by :\\
    Dr. Saurabh Gupta
\end{flushright}
\rule{\textwidth}{1pt}
\begin{abstract}



\end{abstract}
\rule{\textwidth}{1pt}
\newpage
\textbf{TABLE OF CONTENT}\\
\rule{\textwidth}{1pt}
\tableofcontents
\rule{\textwidth}{1pt}

\section{Introduction}
The process of detecting anomalies, patterns, and correlations within massive databases in order to forecast future outcomes is known as data mining. This is accomplished by combining three fields that are intertwined: statistics, artificial intelligence, and machine learning. Data mining is an automated technique that involves looking for patterns in vast information that humans might miss.


Data mining approaches, for example, are used in weather forecasting. Weather forecasting involves analysing large amounts of past data to find trends and forecast future weather conditions based on the time of year, climate, and other factors.



As a result of this analysis, algorithms or models are developed that collect and analyse data in order to anticipate outcomes with increasing accuracy.

\section{Importance of Data Mining}
Data mining is an important part of every organization's analytics programme. The data it generates can be used in BI and advanced analytics programmes that analyse historical data, as well as real-time analytics systems that look at data as it's being created or collected.

Effective data mining benefits in several elements of business strategy development and operations management. Marketing, advertising, sales, and customer service are examples of client-facing functions, as well as manufacturing, supply chain management, finance, and human resources. Fraud detection, risk management, cybersecurity planning, and a variety of other key corporate use cases are all aided by data mining. Healthcare, government, scientific research, mathematics, athletics, and other fields all benefit from it.
\section{Types of Data Mining}
The main types of data that can be used for data mining are as follows:\\
\\
1. Relational Database:\\
\\
A relational database is a collection of several data sets that are formally organised by tables, records, and columns that may be accessed in a variety of ways without knowing the database tables. Tables let people find and share information, making data search, reporting, and organisation easier.\\
\\
2.Data Warehouses:\\
\\
A Data Warehouse is a piece of software that gathers data from multiple sources within a company in order to deliver useful business insights. The massive volume of data comes from a variety of sources, including Marketing and Finance. The retrieved data is used for analytical purposes and assists a corporate organisation in making decisions. Rather than transaction processing, the data warehouse is intended for data analysis.\\
\\
3.Data Repositories:\\
\\
The Data Repository is a broad term for a data storage location. Many IT professionals, on the other hand, use the phrase to refer to a certain type of arrangement within an IT organisation. For example, a collection of databases in which a company has stored numerous types of data.\\
\\
4.Object-Relational Database:\\
\\
An object-relational model is a hybrid of an object-oriented database model with a relational database model. It supports Classes, Objects, and Inheritance, among other things.One of the main goals of the object-relational data model is to bridge the gap between relational databases and the object-oriented model techniques common in many programming languages, such as C++, Java, and C#.\\
\\
\section{Data Mining Techniques}
To turn enormous amounts of data into meaningful information, data mining employs a variety of algorithms and methodologies. Here are a few of the most popular:

1.Association Rules: The term "association rule" refers to a rule-based method for determining associations between variables in a dataset.  linkages between different items, typically employs these methodologies. Businesses may develop stronger cross-selling strategies and recommendation engines by understanding their customers' consumption habits.

2. Neural Networks: Neural networks, which are mostly used for deep learning algorithms, analyse training data by simulating the interconnection of the human brain through layers of nodes. Inputs, weights, a bias (or threshold), and an output make up each node. If the output value reaches a certain threshold, the node "fires" or "activates," sending data to the network's next layer. Through supervised learning, neural networks learn this mapping function, then alter it based on the loss function using gradient descent. We can be sure in the model's accuracy to produce the correct answer when the cost function is at or near zero.
Market basket analysis, which allows organisations to better understand
\end{document}
